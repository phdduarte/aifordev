### Gabarito Comentado: Frequência de Medicação

A resposta correta é: **24 horas**.

Agora, vamos explicar a questão passo a passo:

#### Tema da Questão

Esta questão aborda o conceito de múltiplos comuns em matemática, especificamente o **mínimo múltiplo comum (MMC)**. O MMC de dois ou mais números é o menor número inteiro positivo que é múltiplo de cada um dos números. Esse conceito é frequentemente usado para sincronizar eventos ou ciclos que ocorrem em diferentes períodos.

##### Conhecimentos Necessários

1. **Múltiplos e Divisores:** Entender quais são os múltiplos de um número.
2. **Fatoração de Números:** Saber fatorar números em fatores primos.
3. **Mínimo Múltiplo Comum:** Compreender como encontrar o MMC de dois ou mais números.

#### Justificativa da Resposta

Para determinar em quantas horas Gustavo tomará os três remédios simultaneamente novamente, é necessário calcular o MMC dos períodos de tempo em que ele toma cada remédio: 4 horas, 6 horas e 8 horas. Este mínimo múltiplo comum nos dará o intervalo de tempo após o qual todos os três remédios coincidirão novamente.

#### Passo a Passo para Resolver Matematicamente

##### Passo 1: Lista de Múltiplos
Primeiro, podemos listar alguns múltiplos de cada número para ter uma visão inicial:
- Múltiplos de 4: \(4, 8, 12, 16, 20, 24, \dots\)
- Múltiplos de 6: \(6, 12, 18, 24, \dots\)
- Múltiplos de 8: \(8, 16, 24, \dots\)

Observamos que o número 24 aparece nas três listas, indicando que ele é um múltiplo comum. Vamos verificar isso matematicamente usando a técnica de fatoração.

##### Passo 2: Fatoração em Números Primos

Fatoramos cada período:
- **4:** \(2^2\)
- **6:** \(2^1 \times 3^1\)
- **8:** \(2^3\)

##### Passo 3: Determinação dos Fatores Máximos
Para encontrar o MMC, tomamos o maior expoente de cada fator primo presente nas decomposições:
- O maior expoente de \(2\) é \(2^3\) (adequado ao 8).
- O maior expoente de \(3\) é \(3^1\) (adequado ao 6).

O MMC é então o produto desses fatores com os maiores expoentes:
\[ MMC = 2^3 \times 3^1 \]
\[ MMC = 8 \times 3 = 24 \]

Portanto, Gustavo tomará os três remédios juntos novamente após 24 horas.

#### Explicação Conclusiva

Ao calcular os múltiplos e aplicar a técnica de fatoração, ficou claro que a menor quantidade de horas em que os três ciclos de medicação coincidem é 24 horas. Este resultado é intuitivamente corroborado pelos múltiplos listados e confirmado pela abordagem matemática rigorosa da determinação do MMC através da fatoração.
